% Options for packages loaded elsewhere
\PassOptionsToPackage{unicode}{hyperref}
\PassOptionsToPackage{hyphens}{url}
%
\documentclass[
]{article}
\usepackage{amsmath,amssymb}
\usepackage{lmodern}
\usepackage{iftex}
\ifPDFTeX
  \usepackage[T1]{fontenc}
  \usepackage[utf8]{inputenc}
  \usepackage{textcomp} % provide euro and other symbols
\else % if luatex or xetex
  \usepackage{unicode-math}
  \defaultfontfeatures{Scale=MatchLowercase}
  \defaultfontfeatures[\rmfamily]{Ligatures=TeX,Scale=1}
\fi
% Use upquote if available, for straight quotes in verbatim environments
\IfFileExists{upquote.sty}{\usepackage{upquote}}{}
\IfFileExists{microtype.sty}{% use microtype if available
  \usepackage[]{microtype}
  \UseMicrotypeSet[protrusion]{basicmath} % disable protrusion for tt fonts
}{}
\makeatletter
\@ifundefined{KOMAClassName}{% if non-KOMA class
  \IfFileExists{parskip.sty}{%
    \usepackage{parskip}
  }{% else
    \setlength{\parindent}{0pt}
    \setlength{\parskip}{6pt plus 2pt minus 1pt}}
}{% if KOMA class
  \KOMAoptions{parskip=half}}
\makeatother
\usepackage{xcolor}
\IfFileExists{xurl.sty}{\usepackage{xurl}}{} % add URL line breaks if available
\IfFileExists{bookmark.sty}{\usepackage{bookmark}}{\usepackage{hyperref}}
\hypersetup{
  hidelinks,
  pdfcreator={LaTeX via pandoc}}
\urlstyle{same} % disable monospaced font for URLs
\usepackage{longtable,booktabs,array}
\usepackage{calc} % for calculating minipage widths
% Correct order of tables after \paragraph or \subparagraph
\usepackage{etoolbox}
\makeatletter
\patchcmd\longtable{\par}{\if@noskipsec\mbox{}\fi\par}{}{}
\makeatother
% Allow footnotes in longtable head/foot
\IfFileExists{footnotehyper.sty}{\usepackage{footnotehyper}}{\usepackage{footnote}}
\makesavenoteenv{longtable}
\setlength{\emergencystretch}{3em} % prevent overfull lines
\providecommand{\tightlist}{%
  \setlength{\itemsep}{0pt}\setlength{\parskip}{0pt}}
\setcounter{secnumdepth}{-\maxdimen} % remove section numbering
\ifLuaTeX
  \usepackage{selnolig}  % disable illegal ligatures
\fi

\author{}
\date{}

\begin{document}

\textbf{Polygenic signatures of psychiatric drug response{~}}

Patient response to prescription medication is often highly
unpredictable, ranging from total remission to life-threatening side
effects. Response to treatment depends on a variety of genetic,
environmental, and lifestyle factors that are often beyond the view of
the prescribing physician. Pharmacogenomics is a research field
dedicated to uncovering the genetic basis of inter-individual
variability in drug response, with the aim of providing personalized
interventions. However, currently available pharmacogenomic services are
too narrow in their ascertainment of genetic variation to make robust
personalized drug recommendations. There is an emphasis on metabolism
and known drug targets and mechanisms of action that leads to blind
spots relating to both off-target effects as well as additional
therapeutic mechanisms. A more holistic integration of genomic
information could provide recommendations that maximize therapeutic
effect while minimizing side effects.{~}

We propose to address this critical barrier in pharmacogenomics by
integrating whole-genome genotype information, expression quantitative
trait loci (eQTL), genome-wide association studies (GWAS) of psychiatric
traits, and transcriptional perturbation assays that use small molecules
as perturbagens. The triangulation of these disparate data sources will
allow us to build models with reduced bias, and these models will be
able to recommend drugs on an individual basis that ``normalize''
disease-associated transcriptional signatures. We will also develop new
tools, in the form of a new word embedding space, that capture the
subjective experience of taking psychoactive medication.{~}

With my strong data analysis and programming background, and under the
mentorship of experienced computational biologist Dr. Jacob Michaelson,
I am well-positioned to carry out the proposed research on polygenic
predictors of drug response.{~}

Our proposed approach will integrate genetic, brain gene expression,
behavioral, and subjective data to provide a more comprehensive
understanding of drug response and improve personalized drug
recommendations. We propose to accomplish this overall goal through the
following specific aims:{~}

\textbf{Aim 1:} To enable \textbf{personalized drug recommendation}, we
will integrate individual-level genotype data with polygenic scores and
drug perturbation signatures. We will recommend drugs based on their
tendency to ``normalize'' the disease-associated components of an
individual's imputed (using genotype and eQTL) gene expression
signature. The recommendation will be based on a consensus of 1)
normalizing the individual's gene expression signature while also 2)
normalizing the overall gene expression signature associated with the
psychiatric trait of interest.{~}

\textbf{Aim 2:} To \textbf{identify spatial patterns in susceptibility}
to small molecules in the brain, we will integrate high-resolution brain
gene expression data and trait maps with drug perturbation signatures.
The proposed brain susceptibility map will link specific compounds to
their anticipated phenotypic effects based on publicly available
meta-analytic brain-trait maps. The map will be used to further
prioritize (or de-prioritize) drug recommendations based on anticipated
effects in specific brain regions.{~}

\textbf{Aim 3:} To systematically \textbf{characterize the subjective
experience} of taking specific psychoactive compounds, we will develop a
word embedding space derived from first-person accounts of experience
with selected drugs. The proposed approach will identify recurrent
patterns and trends in drug-induced effects on the mind, which can be
difficult to estimate using objective measures alone.{~}

Together, the successful achievement of these independent aims will
provide new tools to better personalize drug recommendation and
prescription. Our preliminary data suggest that individual genotype is a
powerful tool in this effort and can provide promising results for drug
recommendation.{~}

Table of Contents

1. Introduction{ }4

1.1. Background and Significance{ }4

1.2. Research Question{ }5

Objectives of the study{ }5

1.3. Hypothesis{ }5

2. Literature Review{ }6

2.1. overview of pharmacogenomics and personalized medicine{ }6

2.2. Gene expression, eQTL, and GWAS related to psychiatric disorders:{
}7

2.3. Drug repositioning{ }9

2.4. Psychoactive compounds and subjective experience{ }11

3. Aims Overview/Methods{ }12

3.1. Aim 1:{ }12

3.1.1. eQTL weights and transcriptome imputation{ }13

3.1.2. Chemical perturbagens signature and drug response{ }13

Aim 2:{ }13

Aim 3:{ }13

Proposal Timeline{ }13

\textbf{}~\\

1. Introduction

1.1. Background and Significance

Genetics has been one of the most growing science fields, and yet we
still know too little about diseases mechanisms or be able to provide
proper treatments. Individual's genetics has been known to be a major
player in differentiating response to different pharmacological
compounds. The field of pharmacogenomics is rapidly growing, with the
aim of providing personalized interventions for patients by uncovering
the genetic basis of inter-individual variability in drug response.
However, currently available pharmacogenomic services are limited in
their ascertainment of genetic variation, with an emphasis on metabolism
and known drug targets and mechanisms of action. This approach may
result in blind spots relating to off-target effects as well as
additional therapeutic mechanisms. A more comprehensive understanding of
drug response could provide recommendations that maximize therapeutic
effect while minimizing side effects.

Mental disorders are a major challenge for individuals and society.
Conditions such as schizophrenia, major depression, and anxiety
disorders require long-term treatment with psychoactive drugs. Although
there have been more than two-hundred drugs developed in the last six
decades, they still can have variable effects between patients due to
differences in drug metabolism and action. Consequently, increasing the
dosage of medication does not necessarily lead to better treatment
outcomes. Early treatment outcomes are frequently poor, with 30--50\% of
patients failing first-line antidepressant medication due to
inefficiency or intolerance, according to estimates (Barak et al.,
2011). Moreover, about 25,000 people each year in the United States
visit emergency rooms as a result of side effects brought on by
antidepressants (Hampton et al., 2014). Before identifying a medication
that reduces depressive symptoms with few side effects, patients
frequently try with a variety of antidepressant regimens. The
psychological and societal costs of repeatedly taking drugs that "do not
work" can be distressing for the individual and highlight the need for
better drug selection and dosage tactics because antidepressant
pharmacotherapy studies frequently take a minimum of 6-8 weeks.

In particular, there is a critical need for personalized drug
recommendations for patients with psychiatric disorders. These patients
often have complex medication regimens and can experience a wide range
of responses to treatment, from total remission to life-threatening side
effects. By integrating whole-genome genotype information, expression
quantitative trait loci (eQTL), Polygenics Scores (PGS) of psychiatric
traits, and transcriptional perturbation assays, it may be possible to
develop models with reduced bias that can recommend drugs on an
individual basis.

This research proposal aims to address the critical need for
personalized drug recommendations for patients with psychiatric
disorders by integrating disparate data sources to develop more
comprehensive models of drug response. The proposed research builds on
previous studies that have identified genetic and transcriptional
signatures associated with psychiatric disorders, as well as studies
that have investigated the effects of small molecules on gene
expression. By integrating these data sources, we aim to develop a more
comprehensive understanding of drug response that takes into account the
complex genetic and environmental factors that contribute to
inter-individual variability.

The significance of this research lies in its potential to improve
patient outcomes by providing personalized drug recommendations that
maximize therapeutic effect while minimizing side effects. By developing
more comprehensive models of drug response that consider the complex
genetic and environmental factors that contribute to inter-individual
variability, we hope to provide clinicians with a powerful tool for
optimizing patient care. Ultimately, the proposed research could lead to
improved treatment outcomes and a better quality of life for patients
with psychiatric disorders.

1.2. Research Question

This study aims to address the following research questions:

\begin{itemize}
\tightlist
\item
  Can integrating whole-genome genotype information, expression
  quantitative trait loci (eQTL), and transcriptional perturbation
  assays improve the accuracy of personalized drug recommendations?
\item
  Can the integration of high-resolution brain gene expression data and
  drug perturbation signatures help identify spatial patterns in
  susceptibility to small molecules in the brain, and can these patterns
  be used to further prioritize or de-prioritize drug recommendations
  based on anticipated effects in specific brain regions?
\item
  Can the development of a new word embedding space capture the
  subjective experience of taking psychoactive medication, and can it be
  used to improve personalized drug recommendations?
\end{itemize}

Objectives of the study

\emph{this is redundant with the section above, probably only keep one?}

The objectives of this study are to:

\begin{itemize}
\tightlist
\item
  Integrate whole-genome genotype information, eQTL, GWAS, and
  transcriptional perturbation assays to recommend drugs on an
  individual basis that ``normalize'' disease-associated transcriptional
  signatures.
\item
  Integrate high-resolution brain gene expression data and trait maps
  with drug perturbation signatures to identify spatial patterns in
  susceptibility to small molecules in the brain and use them to further
  prioritize or de-prioritize drug recommendations based on anticipated
  effects in specific brain regions.
\item
  Develop a new word embedding space that captures the subjective
  experience of taking psychoactive medication and uses it to improve
  personalized drug recommendations.
\end{itemize}

1.3. Hypothesis

This study is based on the following hypotheses:

\begin{itemize}
\tightlist
\item
  Integration of whole-genome genotype information and eQTL data is
  sufficient enough to predict gene expression in different brain
  tissue, which could be integrated with transcriptional perturbation
  assays to predict personalized drug response.
\item
  Integration of high-resolution brain gene expression data with drug
  perturbation signatures can efficiently provide a gradient of drug
  activity in the brain. The drug gradient is hypothesized to be
  equivalent to anticipated effects in specific brain regions, that
  could be related to drug side effects.
\item
  Psychoactive medications are more likely to cause different
  experiences that could not be captured by objective measures. We are
  hypothesizing that there is a highly similar subjective experiences
  among patients taking the same drug and could not be measured through
  the first two approaches.
\end{itemize}

2. Literature Review

2.1. overview of pharmacogenomics and personalized medicine

Pharmacogenomics, also known as pharmacogenetics, is the branch of
science that looks at how a person's genes influence how they react to
pharmaceuticals. Its long-term objective is to assist physicians in
choosing the medications and dosages that are ideal for every patient.
It falls under the category of precision medicine, which tries to treat
every patient uniquely.

Pharmacogenomics, a rapidly developing field in personalized medicine,
aims to elucidate how variations in genes can affect a patient's
response to drugs. The influence of genes on drug metabolism and
efficacy is well-established, as they encode for enzymes and proteins
responsible for the breakdown and uptake of medications in the body.

Of particular interest are genes that encode for enzymes involved in
drug metabolism, such as CYP2D6, which acts on a quarter of all
prescription drugs. Multiple variations of this gene exist, with some
individuals having multiple copies of it. These genetic variations can
result in differences in enzyme activity, with some variants leading to
a hyperactive enzyme that metabolizes drugs at a faster rate than normal
(Ingelman-Sundberg, 2005). This can result in drug overdose,
particularly in the case of codeine, which is metabolized by CYP2D6 to
produce its active form, morphine. Conversely, some variants of CYP2D6
produce an enzyme that is non-functional or less active, leading to
reduced or absent drug efficacy.

Therefore, understanding the impact of genetic variations on drug
response is crucial in ensuring safe and effective drug therapy.
Pharmacogenomics provides a valuable tool for predicting drug response
based on an individual's genetic makeup, enabling personalized medicine
approaches for improved patient outcomes.

The FDA has released comments and warning letters on pharmacogenetic
testing in response to concerns over the marketing of these tests. The
efficacy of clinical pharmacogenetic testing may not be fully supported
by clinical data, according to a safety communication released on
November 01, 2018. The statement in this safety communication that "the
relationship between DNA variations and the effectiveness of
antidepressant medication has never been established" particularly
emphasized the use of pharmacogenetic testing to guide antidepressant
drug prescribing (Shuren, 2018). In keeping with the FDA goal to
safeguard and advance the public's health, it's critical to act right
away to make sure that the claims being made about the pharmacogenetic
tests currently available are supported by reliable research. That can
be done by taking measures that safeguard patients while also advancing
the creation of analytically and clinically validated pharmacogenetic
tests. Recently, the FDA released a new web-based resource that includes
a table including some of the pharmacogenetic associations with a last
update on October 26, 2022 (FDA, 2022). Some of these have detailed
information regarding therapeutic management, but the majority of the
associations listed have not been assessed in terms of the effect of
genetic testing on clinical outcomes, such as improved therapeutic
effectiveness or increased risk of particular adverse events. This
version of the table is restricted to pharmacogenetic associations
linked to drug transporter, drug metabolizing enzyme, and gene
variations associated with a susceptibility for certain adverse
outcomes.

Pharmacogenomic testing could be done in two forms: single-gene or a
multi-panel testing. Most of the developed pharmacogenetic tests
investigate the main drug gene targets, and the advanced ones with a
multigene panel investigate different genes that include the ones
involved in the process of metabolizing the drug. Yet, none of these
tests consider a whole-genome approach, which is one of our main focuses
in the proposed study. one of the main advantages of the proposed
approach is that we are focusing on the entire genome, not only the
genes involved in a drug response or metabolism. The FDA now advises
against using direct-to-consumer testing for making medical choices,
even though there has been considerable debate about the use of
pharmacogenomics in clinical practice. However, the FDA allowed the
marketing of the 23andMe Personal Genome Service Pharmacogenetic Reports
test as a direct-to-consumer test with special controls for informing
discussions with a healthcare professional about genetic variants that
may be related to a patient's capacity to metabolize some medications
(FDA, 2018).

2.2. Gene expression, eQTL, and GWAS related to psychiatric disorders:

In recent years, a new area of research has emerged: the investigation
of the genetic basis of psychiatric traits. The use of eQTLs, GWAS, and
gene expression data is one strategy that has gained interest in this
field. Genomic regions known as eQTLs are linked to variations in gene
expression levels. Finding eQTLs allows researchers to uncover genetic
variations that may have an effect on the transcriptome functionally,
perhaps shedding light on the molecular processes behind mental
illnesses. For instance, eQTLs for numerous genes, including ITIH4,
GLT8D1, GNL3, and NEK4, were discovered to be enriched in
schizophrenia-related genetic regions (Kim et al., 2014). This shows
that these genes could be involved in the disorder's genesis. eQTLs were
also studied for their effects in the developing human brain and their
enrichment in neuropsychiatric disorders (Bryois et al., 2022; O'Brien
et al., 2018).

The use of GWAS to investigate the genetic foundation of psychiatric
characteristics has also become widespread. In these studies, the
complete genome is examined in sizable patient cohorts and healthy
controls, revealing genetic variations linked to certain disorders. For
instance, a recent meta-analysis of three major genome-wide association
study (GWAS) of major depressive disorder (MDD) discovered 102
independent variants, 15 gene-sets, and 269 genes to be linked to the
condition, many of which are involved in synaptic neurotransmission and
synaptic structure (Howard et al., 2019). The functional implications of
these polymorphisms are not revealed by GWAS, even though they serve as
a useful starting point for the identification of genetic risk factors.
The value of these association analysis techniques is increased when
evidence of biological pathway enrichment is combined with data on
features associated with the trait being studied. This allows for
further conclusions regarding similar aetiological processes.

The analysis of different genome variants in the context of their effect
on gene has spawned a big field in genetics named as expression
quantitative trait loci (eQTLs). An eQTL is a locus that explains a
portion of a gene expression phenotype's genetic variation. In a
standard eQTL experiment, gene expression levels are often assessed in
tens or hundreds of people and genetic variation markers are directly
tested for associations. This association study can be performed close
to the gene (cis) or far (trans) from it. The \emph{cis}-eQTLs
variations are those that are within 1 Mb (megabase) on either side of a
gene's transcription start site (TSS). On the other hand,
\emph{trans}-eQTLs are those that are at least 5 Mb upstream,
downstream, or on a separate chromosome. The same regulatory region or
variant can be identified as an eQTL for different genes in different
tissues, suggesting that tissue specificity is a highly important factor
to be considered in such analyses. {~}

\hfill\break

Figure 1 (A) shows a schematic representation of the effect of an eQTL
on gene expression. Boxplots in (B) show correlation between genotype
variants and gene expression in different populations. The width of a
boxplot is proportional to the allele frequency.

2.3. Drug repositioning{~}

Drug repurposing initiatives may benefit from the application of GWAS
and eQTL analyses. eQTL analysis can show the functional impact of these
variations on gene expression, and GWAS can find genetic variants linked
to a specific disease or health issue. Researchers can find medications
that may be beneficial for treating diseases other than their intended
targets by merging these datasets with drug databases.{~}

A recent article reviewed different methods for drug repurposing that
included 5 main approaches utilizing different types of datasets, shown
in Figure 1 (Lau \& So, 2020).{~ }In the first approach ``candidate gene
approach'', which uses functional annotations and eQTL data, the risk
loci from GWAS data may be mapped to the genes that are most likely to
be important. Drugs may be used as repositioning candidates if the
discovered candidate gene is druggable and the medication is not
currently prescribed for the condition. Another approach, called
``pathway or gene-set analysis approach'', states that drugs that target
members of the same pathway are prospective drug candidates. The
pathways that the selected candidate gene(s) are situated in are next
examined. In addition, enrichment tests may be used to find medications
whose targets or effector genes obtain higher significance (lower
p-values) than anticipated overall. The whole collection of GWAS data
may also be utilized to produce gene-based statistics. The third
approach is mainly focusing on comparing similarities between drugs and
diseases (dx). In that approach, if two medications' indications are
sufficiently similar to one another, they can be relocated. For
instance, utilizing cell-line expression data from the Connectivity Map
(CMap), one may assess the similarities between the transcriptomes of
two medications. According to this, if two diseases are similar, then
the medications used to treat one condition may be used to treat the
other. The fourth approach is considered by looking for reversed
expression patterns between drugs and diseases. The key premise is that
a medication may be a candidate for repositioning if it results in an
expression profile that is the opposite of that of a disease (owing to
its ability to "reverse" expression patterns linked to diseases). The
fifth, and last approach, is a network-based analysis approach. The
method is based on building biological networks by integrating data from
several sources, including interactions between drugs, proteins, genes,
and diseases. The idea behind this approach is similar to that of
"similarity-based" methods for medication repositioning, however
network-based approaches often use a wider range of data.

\hfill\break

Figure 2: A schematic diagram of drug repurposing approaches using
different types of datasets reviewed by (Lau \& So, 2020).

2.4. Psychoactive compounds and subjective experience

Psychoactive compounds are those that work on the central nervous system
that impact a person's perception, behavior, and mood. Although
psychoactive substances have been used for ages for religious,
therapeutic, and recreational purposes, their usage can also have
unfavorable effects including addiction, psychosis, and other
psychological issues. Understanding the subjective experience of
consuming psychoactive substances is crucial for producing safer and
more potent psychoactive substances as well as for improving the
treatment results for psychiatric disorders. The effects of psychoactive
substances on the brain and body have been investigated using unbiased
techniques including physiological monitoring and brain imaging. These
metrics, meanwhile, fall short of properly capturing the user's
subjective experience. This is due to the fact that the subjective
experience of ingesting psychoactive substances is a complicated
phenomenon with many facets that includes a variety of cognitive,
emotional, and perceptual processes. As a result, to fully comprehend
the subjective experience of ingesting psychoactive substances, it is
necessary to employ additional techniques that can adequately capture
the subtleties of the effects that drugs have on the brain.

The study of first-person narratives of drug experiences is a potential
strategy for comprehending the subjective experience of taking
psychoactive substances. We can better grasp the variety of effects that
these substances can have as well as the variations in responses across
individuals by examining first-person accounts. It can be difficult to
analyze unstructured data, such as first-person experiences, thus it's
critical to create effective strategies for drawing out useful
information from these experiences. Word embedding techniques have been
utilized more often recently to analyze unstructured text data and
uncover significant trends. A form of natural language processing (NLP)
method known as word embeddings may represent words as high-dimensional
vectors that capture their semantic and contextual links. A word
embedding space on first-person reports of drug experiences can be
developed to represents the subjective experience of taking certain
psychoactive substances. This method can uncover recurring patterns and
trends in the mental side effects of drugs, which can offer insightful
information on the subjective experience of taking psychoactive
substances. By doing this, we want to improve our comprehension of the
intricate and varied phenomena known as the subjective experience of
taking psychoactive substances.

3. Aims Overview/Methods

3.1. Aim 1:{~}

We propose to address the critical barrier in pharmacogenomics by
integrating tissue-specific gene expression data and transcriptional
perturbation assays that use small molecules as perturbagens. The
combination of these data sources will allow us to build a model with
reduced bias, which will be able to recommend drugs on an individual
basis that "normalize" disease-associated transcriptional signatures.
For example, if a person is diagnosed with depression and depression
caused elevation of gene X expression, the best drug to be working well
with this patient is the one that can neutralize the disease effect on
gene expression and bring it back to normal level. Our overall objective
in this work is to maximize efficacy of drugs on an individual level,
while reducing the negative side effects.

3.1.1. eQTL weights and transcriptome imputation

To identify how different variants on the genome affect gene expression
by tissue, the eQTL data from the Genotype-Tissue Expression (GTEx)
project will be used to give weights for every variant by gene. The
weights are mainly derived from a linear regression model that fits a
gene expression in a specific tissue by the known cis variants, close by
a 1 Mb distance from the TSS. The linear regression model comes in the
form of:{~}

{~}(1)

where B1:Bn are the effect sizes that each SNP X has on that gene, and n
is the total number of cis-eQTLs of the given gene. By the end of this
step, we will have a weights matrix of MxN for every single tissue,
where N is the number of genes, and M is the number of SNPs known to
have weights for any gene in that tissue. If the SNP is not affecting
the gene's expression, the value will be 0. Otherwise, it will have the
weight derived from the linear model.{~}

The step of identifying the effect sizes of \emph{cis}-eQTLs per gene in
every tissue from GTEx was done different groups of researchers with
different approaches (Chen et al., 2023; de Klein et al., 2023; Gamazon
et al., 2015; Hu et al., 2019; Liu \& Kang, 2022). The model weights
from (Hu et al., 2019) will be used for transcriptome imputation in this
project.{~}

To impute individuals' gene expression in different tissues, a simple
matrix multiplication will be done following this formula:{~}

{~}(2)

The \emph{Tissue\_Weights\_Matrix} represents the weights matrix, MxN,
derived from equation (1). On the other hand, the
\emph{Genotypes\_Matrix} has rows as individuals, and columns as
genotypes. The values in this matrix represent the genotype alleles type
found per individuals (i.e., 0 if both alleles are homozygous reference,
1 if both alleles are heterozygous, and 2 if both alleles are homozygous
alternative). The \emph{Genotypes\_Matrix} used in this project is for
participants from the Adolescent Brain Cognitive Development (ABCD)
study. The ABCD study is one of the largest long-term NIH-funded study
of the brain development and child health in the United States. The
study includes \textasciitilde11,000 children of ages 9-10 years. The
study tracks their biological and behavioral development through
adolescence into young adulthood.{~}

3.1.2. Chemical perturbagens signature and drug response

\hfill\break

Aim 2:

\hfill\break

Aim 3:

\hfill\break

Proposal Timeline

{~}

\begin{longtable}[]{@{}
  >{\raggedright\arraybackslash}p{(\columnwidth - 16\tabcolsep) * \real{0.11}}
  >{\raggedright\arraybackslash}p{(\columnwidth - 16\tabcolsep) * \real{0.11}}
  >{\raggedright\arraybackslash}p{(\columnwidth - 16\tabcolsep) * \real{0.11}}
  >{\raggedright\arraybackslash}p{(\columnwidth - 16\tabcolsep) * \real{0.11}}
  >{\raggedright\arraybackslash}p{(\columnwidth - 16\tabcolsep) * \real{0.11}}
  >{\raggedright\arraybackslash}p{(\columnwidth - 16\tabcolsep) * \real{0.11}}
  >{\raggedright\arraybackslash}p{(\columnwidth - 16\tabcolsep) * \real{0.11}}
  >{\raggedright\arraybackslash}p{(\columnwidth - 16\tabcolsep) * \real{0.11}}
  >{\raggedright\arraybackslash}p{(\columnwidth - 16\tabcolsep) * \real{0.11}}@{}}
\toprule
\endhead
Aim & Task & 2022 & 2023 & 2024 & 2025 & & & \\
1 & get weights of eQTLs & & & & & & & \\
& impute gene expression for samples & & & & & & & \\
& predict individuals' drug response & & & & & & & \\
& validate results using CBCL & & & & & & & \\
2 & build a model for predicting gene expression & & & & & & & \\
& compute drug activity in brain regions & & & & & & & \\
& correlate drug activity with brain functionality from neurosynth & & &
& & & & \\
& apply same model on mouse brain and validate & & & & & & & \\
3 & download and preprocess text data of users of interest & & & & & &
& \\
& build a word embedding space for experience reports & & & & & & & \\
& validate results and reflect on aim 2 brain maps & & & & & & & \\
\bottomrule
\end{longtable}

\hfill\break

\hfill\break

\hfill\break

\hfill\break

\hfill\break

\emph{points to be mentioned:}{~}

\begin{itemize}
\tightlist
\item
  {Therapeutic drug monitoring (TDM) has emerged as a promising solution
  to these challenges, particularly for mood stabilizers,
  antidepressants, and antipsychotics. TDM has the potential to reduce
  variability, speed up clinical improvement, and improve drug
  tolerability and safety.}{~}
\item
  Link the third aim to the first two aims
\item
  Make a graphical abstract
\item
\end{itemize}

References

\hfill\break

Barak, Y., Swartz, M., \& Baruch, Y. (2011). Venlafaxine or a second
SSRI: Switching after treatment failure with an SSRI among depressed
inpatients: a retrospective analysis. \emph{Prog Neuropsychopharmacol
Biol Psychiatry}, \emph{35}(7), 1744-1747.
https://doi.org/10.1016/j.pnpbp.2011.06.007{~}

Bryois, J., Calini, D., Macnair, W., Foo, L., Urich, E., Ortmann, W.,
Iglesias, V. A., Selvaraj, S., Nutma, E., Marzin, M., Amor, S.,
Williams, A., Castelo-Branco, G., Menon, V., De Jager, P., \& Malhotra,
D. (2022). Cell-type-specific cis-eQTLs in eight human brain cell types
identify novel risk genes for psychiatric and neurological disorders.
\emph{Nat Neurosci}, \emph{25}(8), 1104-1112.
https://doi.org/10.1038/s41593-022-01128-z{~}

Chen, F., Wang, X., Jang, S. K., Quach, B. C., Weissenkampen, J. D.,
Khunsriraksakul, C., Yang, L., Sauteraud, R., Albert, C. M., Allred, N.
D. D., Arnett, D. K., Ashley-Koch, A. E., Barnes, K. C., Barr, R. G.,
Becker, D. M., Bielak, L. F., Bis, J. C., Blangero, J., Boorgula, M. P.,
. . . Liu, D. J. (2023). Multi-ancestry transcriptome-wide association
analyses yield insights into tobacco use biology and drug repurposing.
\emph{Nat Genet}, \emph{55}(2), 291-300.
https://doi.org/10.1038/s41588-022-01282-x{~}

de Klein, N., Tsai, E. A., Vochteloo, M., Baird, D., Huang, Y., Chen, C.
Y., van Dam, S., Oelen, R., Deelen, P., Bakker, O. B., El Garwany, O.,
Ouyang, Z., Marshall, E. E., Zavodszky, M. I., van Rheenen, W., Bakker,
M. K., Veldink, J., Gaunt, T. R., Runz, H., . . . Westra, H. J. (2023).
Brain expression quantitative trait locus and network analyses reveal
downstream effects and putative drivers for brain-related diseases.
\emph{Nat Genet}, \emph{55}(3), 377-388.
https://doi.org/10.1038/s41588-023-01300-6{~}

FDA. (2018, October 31). \emph{FDA authorizes first direct-to-consumer
test for detecting genetic variants that may be associated with
medication metabolism}
https://www.fda.gov/news-events/press-announcements/fda-authorizes-first-direct-consumer-test-detecting-genetic-variants-may-be-associated-medication

FDA. (2022). Table of Pharmacogenetic Associations. In.

Gamazon, E. R., Wheeler, H. E., Shah, K. P., Mozaffari, S. V.,
Aquino-Michaels, K., Carroll, R. J., Eyler, A. E., Denny, J. C.,
Consortium, G. T., Nicolae, D. L., Cox, N. J., \& Im, H. K. (2015). A
gene-based association method for mapping traits using reference
transcriptome data. \emph{Nat Genet}, \emph{47}(9), 1091-1098.
https://doi.org/10.1038/ng.3367{~}

Hampton, L. M., Daubresse, M., Chang, H. Y., Alexander, G. C., \&
Budnitz, D. S. (2014). Emergency department visits by adults for
psychiatric medication adverse events. \emph{JAMA Psychiatry},
\emph{71}(9), 1006-1014.
https://doi.org/10.1001/jamapsychiatry.2014.436{~}

Howard, D. M., Adams, M. J., Clarke, T. K., Hafferty, J. D., Gibson, J.,
Shirali, M., Coleman, J. R. I., Hagenaars, S. P., Ward, J., Wigmore, E.
M., Alloza, C., Shen, X., Barbu, M. C., Xu, E. Y., Whalley, H. C.,
Marioni, R. E., Porteous, D. J., Davies, G., Deary, I. J., . . .
McIntosh, A. M. (2019). Genome-wide meta-analysis of depression
identifies 102 independent variants and highlights the importance of the
prefrontal brain regions. \emph{Nat Neurosci}, \emph{22}(3), 343-352.
https://doi.org/10.1038/s41593-018-0326-7{~}

Hu, Y., Li, M., Lu, Q., Weng, H., Wang, J., Zekavat, S. M., Yu, Z., Li,
B., Gu, J., Muchnik, S., Shi, Y., Kunkle, B. W., Mukherjee, S.,
Natarajan, P., Naj, A., Kuzma, A., Zhao, Y., Crane, P. K., Alzheimer's
Disease Genetics, C., . . . Zhao, H. (2019). A statistical framework for
cross-tissue transcriptome-wide association analysis. \emph{Nat Genet},
\emph{51}(3), 568-576. https://doi.org/10.1038/s41588-019-0345-7{~}

Ingelman-Sundberg, M. (2005). Genetic polymorphisms of cytochrome P450
2D6 (CYP2D6): clinical consequences, evolutionary aspects and functional
diversity. \emph{Pharmacogenomics J}, \emph{5}(1), 6-13.
https://doi.org/10.1038/sj.tpj.6500285{~}

Kim, Y., Xia, K., Tao, R., Giusti-Rodriguez, P., Vladimirov, V., van den
Oord, E., \& Sullivan, P. F. (2014). A meta-analysis of gene expression
quantitative trait loci in brain. \emph{Transl Psychiatry},
\emph{4}(10), e459. https://doi.org/10.1038/tp.2014.96{~}

Lau, A., \& So, H. C. (2020). Turning genome-wide association study
findings into opportunities for drug repositioning. \emph{Comput Struct
Biotechnol J}, \emph{18}, 1639-1650.
https://doi.org/10.1016/j.csbj.2020.06.015{~}

Liu, A. E., \& Kang, H. M. (2022). Meta-imputation of transcriptome from
genotypes across multiple datasets by leveraging publicly available
summary-level data. \emph{PLoS Genet}, \emph{18}(1), e1009571.
https://doi.org/10.1371/journal.pgen.1009571{~}

O'Brien, H. E., Hannon, E., Hill, M. J., Toste, C. C., Robertson, M. J.,
Morgan, J. E., McLaughlin, G., Lewis, C. M., Schalkwyk, L. C., Hall, L.
S., Pardinas, A. F., Owen, M. J., O'Donovan, M. C., Mill, J., \& Bray,
N. J. (2018). Expression quantitative trait loci in the developing human
brain and their enrichment in neuropsychiatric disorders. \emph{Genome
Biol}, \emph{19}(1), 194. https://doi.org/10.1186/s13059-018-1567-1{~}

Shuren, J. (2018). \emph{Jeffrey Shuren, M.D., J.D., director of the
FDA's Center for Devices and Radiological Health and Janet Woodcock,
M.D., director of the FDA's Center for Drug Evaluation and Research on
agency's warning to consumers about genetic tests that claim to predict
patients' responses to specific medications}. Retrieved March 27 from
https://www.fda.gov/news-events/press-announcements/jeffrey-shuren-md-jd-director-fdas-center-devices-and-radiological-health-and-janet-woodcock-md

\hfill\break

\end{document}
